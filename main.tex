% -*- flyspell-dict: "german" -*-

\documentclass[xcolor=dvipsnames]{beamer}
\mode<presentation>{
  \usetheme{main}
}

\usepackage[ngerman]{babel}
\usepackage[utf8x]{inputenc}
\usepackage{times}
\usepackage[T1]{fontenc}
\usepackage{textcomp}
\usepackage{graphicx}
\usepackage{listings}
\usepackage{semantic}
\usepackage{multicol}
\usepackage{hyperref}
\reservestyle{\command}{\textbf}
\command{accept, request, in}
\bibliographystyle{alphadin}

% ----------------------------------------------------------
% title page:

\title{Infrastrukturkosten Stadt- und Straßenbahnen}

\author[Brandes, Müller, Riecken, Sigler, Sulfrian (Gruppe 7)]{Yves~Müller \and
  Marian~Sigler \and
  Alexander~Sulfrian \and
  Dario~Lino~Brandes \and
  Robert~Hanns~Riecken}
\institute{\\ \vspace{1em}
  Institut für Land- und Seeverkehr\\
  Fachgebiet Schienenfahrwege und Bahnbetrieb\\
  Technische Universität Berlin}

\day=28 \month=05 \year=2013
\date{\today}

% ----------------------------------------------------------

\begin{document}

% ----------------------------------------------------------

\begin{frame}
  \titlepage
\end{frame}

% ----------------------------------------------------------

\begin{frame}
  \frametitle{Inhalt}

  \setcounter{tocdepth}{2}
  \tableofcontents
\end{frame}

% ----------------------------------------------------------

\section{Definitionen}
\begin{frame}
  \frametitle{Definitionen}

  \begin{itemize}
    \item nur Bahnen nach BOStrab
    \item Kategorisierung:
      \begin{itemize}
        \item U-Bahn bzw. Hochbahn
        \item Straßenbahn
      \end{itemize}
    \item Differenzierung von Streckenabschnitten in Tunnel
      oder auf Brücken
  \end{itemize}
\end{frame}

% ----------------------------------------------------------

\section{Vorgehen}
\begin{frame}
  \frametitle{Vorgehen}

  \begin{itemize}    
  \item Ermittlung Daten für div. Projekte:
    \begin{itemize}
    \item geplante/tatsächliche Kosten
    \item Strecken-/Gleislänge
    \item Anzahl Ausweichstellen/Haltstellen
    \item Tunnel-/Brücken- und Hochbahnanteil
    \item und weitere
    \end{itemize}
  \item Umrechnung und Inflationsbereinigung
  \item Berechnung von Durchschnittskosten
  \end{itemize}
\end{frame}


\section{Vorstellung ausgewählter Strecken}
% ----------------------------------------------------------

\subsection{Darmstadt}
\begin{frame}
  \frametitle{Darmstadt (1897)}

  \begin{itemize}
    \item Länge: 6,4km eingleisig mit 8 Ausweichen
      \begin{itemize}
        \item 5,5 km eingleisig
        \item 8 Ausweichen
        \item 0,9 km zweigleisig
      \end{itemize}
    \item Baukosten: 287.000 Mark (5,1 Mio. €)
      \begin{itemize}
        \item Preis pro km: 0,7 Mio. €/km\\
          (2-gleisige Abschnitte doppelt gewichtet)
      \end{itemize}
  \end{itemize}
\end{frame}

% ----------------------------------------------------------

\begin{frame}
  \frametitle{Darmstadt (1899 bis 1903)}

  \begin{itemize}
    \item Diverse Neubaustrecken, Gesamtlänge: 5,98 km
      \begin{itemize}
        \item 5,34 km eingleisig
        \item 4 Ausweichen
        \item 0,64 km zweigleisig
      \end{itemize}
    \item Gesamtkosten: 1,2 Mio. Mark (inklusive Fahrzeuge; etwa 50\%
      davon für Gleisbau)
      \begin{itemize}
        \item Umgerechnet ca. 5,6 Mio. €
        \item Preis pro km: 0,85 Mio. €/km (0,09 Mio. Mark/km)\\
          (2-gleisige Abschnitte doppelt gewichtet)
      \end{itemize}
  \end{itemize}
\end{frame}

% ----------------------------------------------------------

\subsection{Bergen}
\begin{frame}
  \frametitle{Bergen (2010)}

  \begin{itemize}
    \item Länge: 9,8 km
      \begin{itemize}
        \item 4 Tunnel (2,63 km)
        \item Gesamte Strecke 2-gleisig
      \end{itemize}
    \item Baukosten:
      \begin{itemize}
        \item Insgesamt: 140 Mio. € (projektiert, abzgl. ca. 30
          Mio. € für Fahrzeuge)
        \item Preis pro km: 11,2 Mio. €/km
      \end{itemize}
    \item zwei weitere Ausbaustufen geplant
  \end{itemize}
\end{frame}

% ----------------------------------------------------------

\subsection{Aarhus}
\begin{frame}
  \frametitle{Aarhus (2016)}

  \begin{itemize}
    \item Länge:
      \begin{itemize}
        \item Gesamtlänge: 12,0 km
        \item Keine Tunnelneubauten
        \item Weitestgehend auf vorhanden Eisenbahntrassen
      \end{itemize}
    \item Baukosten:
      \begin{itemize}
        \item Insgesamt: 500 Mio. Kronen
        \item Umgerechnet 65 Mio. € (0,13 €/Krone)
        \item Preis pro km: 5,4 Mio. €/km
      \end{itemize}
  \end{itemize}
\end{frame}

% ----------------------------------------------------------

\subsection{Stuttgart}
\begin{frame}
  \frametitle{Stuttgart-Stammheim (2011)}

  \begin{itemize}
    \item Länge: 2,985 km
      \begin{itemize}
        \item ein Tunnel (1 km)
        \item 7 Haltestellen
        \item ersetzt Straßenbahn
      \end{itemize}
    \item Baukosten: 105 Mio. €
    \begin{itemize}
        \item 35,2 Mio. €/km
    \end{itemize}
  \end{itemize}
\end{frame}

% ----------------------------------------------------------
\subsection{Toulouse}
\begin{frame}
  \frametitle{Straßenbahn Toulouse (2010)}
  \begin{itemize}
    \item Länge: 11 km
      \begin{itemize}
        \item 18 Haltestellen
        \item überirdisch, eine Brücke
      \end{itemize}
    \item Baukosten: 212 Mio. €
    \begin{itemize}
        \item 11,8 Mio. €/km
    \end{itemize}
  \end{itemize}
\end{frame}

% ----------------------------------------------------------

\section{Auswertung}
\begin{frame}
    \frametitle{Auswertung}
    \begin{itemize}
      \item TODO
    \end{itemize}
\end{frame}

% ----------------------------------------------------------

\section{Quellen}
\begin{frame}[allowframebreaks]
  \frametitle{Quellen}

  \nocite{UmrechnungGoldmark}
  \nocite{buernheim1997bahnen}
  \nocite{hoeltge1992hessen}
  \nocite{BybaneBergen}
  \nocite{BergenCurrentStatus}
  \nocite{midttrafik}
  \nocite{UmrechnungKronen}
  \nocite{osm}

  \begin{scriptsize}
    \bibliography{main}
  \end{scriptsize}
\end{frame}

% ----------------------------------------------------------



\end{document}
