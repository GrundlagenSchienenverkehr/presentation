% -*- flyspell-dict: "german" -*-

\documentclass{beamer}
\mode<presentation>{
  \usetheme{Warsaw}
}

\usepackage[ngerman]{babel}
\usepackage[utf8x]{inputenc}
\usepackage{times}
\usepackage[T1]{fontenc}
\usepackage{textcomp}
\usepackage{graphicx}
\usepackage{listings}
\usepackage{semantic}
\usepackage{multicol}
\usepackage{hyperref}
\reservestyle{\command}{\textbf}
\command{accept, request, in}
\bibliographystyle{alphadin}

% ----------------------------------------------------------
% title page:

\title{Infrastrukturkosten Stadt- und Straßenbahnen}

\author[Gruppe 7]{Dario Lino Brandes \and
  Yves Müller \and
  Robert Hanns Riecken \and
  Marian Sigler \and
  Alexander Sulfrian}
\institute{\\ \vspace{1em}
  Institut für Land- und Seeverkehr\\
  Fachgebiet Schienenfahrwege und Bahnbetrieb\\
  Technische Universität Berlin}

\day=28 \month=05 \year=2013
\date{\today}

% ----------------------------------------------------------

\begin{document}

% ----------------------------------------------------------

\begin{frame}
  \titlepage
\end{frame}

% ----------------------------------------------------------

\begin{frame}
  \frametitle{Inhalt}

  \setcounter{tocdepth}{2}
  \tableofcontents
\end{frame}

% ----------------------------------------------------------

\section{Grundlagen}
\begin{frame}
  \frametitle{Grundlagen}

  \begin{itemize}
    \item nicht nach EBO
    \item Trennung U-Bahn/Straßenbahn notwendig:
      \begin{itemize}
        \item U-Bahn bzw. Hochbahn: eigener Schienenkörper
        \item Straßenbahn: Gleise in Straße möglich
      \end{itemize}
    \item Unterscheidung von Streckenabschnitten mit Tunnel oder
      Brücken
  \end{itemize}
\end{frame}

% ----------------------------------------------------------

\section{Darmstadt 1897}
\begin{frame}
  \frametitle{Darmstadt 1897}

  \begin{itemize}
    \item Länge: 6,4km eingleisig mit 8 Ausweichen
      \begin{itemize}
        \item 5,5 km eingleisig
        \item 8 Ausweichen
        \item 0,9 km zweigleisig
      \end{itemize}
    \item Baukosten: 287.000 Mark
      \begin{itemize}
        \item Umgerechnet ca. 5,1 Mio. € (17,82 €/Mark)
        \item Preis pro km: 0,7 Mio. €/km (0,04 Mio. Mark/km)\\
          (2-gleisige Abschnitte doppelt gewichtet)
      \end{itemize}
  \end{itemize}
\end{frame}

% ----------------------------------------------------------

\section{Darmstadt 1899 bis 1903}
\begin{frame}
  \frametitle{Darmstadt 1899 bis 1903}

  \begin{itemize}
    \item Diverse Neubaustrecken, Gesamtlänge: 5,98 km
      \begin{itemize}
        \item 5,34 km eingleisig
        \item 4 Ausweichen
        \item 0,64 km zweigleisig
      \end{itemize}
    \item Gesamtkosten: 1,2 Mio. Mark (inklusive Fahrzeuge; etwa 50\%
      davon für Gleisbau)
      \begin{itemize}
        \item Umgerechnet ca. 5,6 Mio. € (9,35 €/Mark)
        \item Preis pro km: 0,85 Mio. €/km (0,09 Mio. Mark/km)\\
          (2-gleisige Abschnitte doppelt gewichtet)
      \end{itemize}
  \end{itemize}
\end{frame}

% ----------------------------------------------------------

\section{Bergen 2010}
\begin{frame}
  \frametitle{Bergen 2010}

  \begin{itemize}
    \item Länge: 9,8 km
      \begin{itemize}
        \item 4 Tunnel (2,63 km): Fageråstunnel (663 m),
          Slettebakkstunnel (412 m), Fantofttunnel (1107 m), Tveiteråstunnel (443 m)
        \item Gesamte Strecke 2-gleisig
      \end{itemize}
    \item Baukosten:
      \begin{itemize}
        \item Insgesamt: 140 Mio. € (projektiert, abzgl. ca. 12x2,5
          Mio. € für Fahrzeuge)
        \item Preis pro km: 11,2 Mio. €/km
      \end{itemize}
    \item 2 weitere Ausbaustufe
  \end{itemize}
\end{frame}

% ----------------------------------------------------------

\section{Aarhus 2016 (in Planung)}
\begin{frame}
  \frametitle{Aarhus 2016 (in Planung)}

  \begin{itemize}
    \item Länge:
      \begin{itemize}
        \item Gesamtlänge: 12,0 km
        \item Keine Tunnelneubauten
        \item Weitestgehend auf vorhanden Eisenbahntrassen
      \end{itemize}
    \item Baukosten:
      \begin{itemize}
        \item Insgesamt: 500 Mio. Kronen
        \item Umgerechnet 65 Mio. € (0,13 €/Krone)
        \item Preis pro km: 5,4 Mio. €/km
      \end{itemize}
  \end{itemize}
\end{frame}

% ----------------------------------------------------------

\begin{frame}[allowframebreaks]
  \frametitle{Quellen}

  \nocite{UmrechnungGoldmark}
  \nocite{buernheim1997bahnen}
  \nocite{hoeltge1992hessen}
  \nocite{BybaneBergen}
  \nocite{BergenCurrentStatus}
  \nocite{midttrafik}
  \nocite{UmrechnungKronen}

  \begin{scriptsize}
    \bibliography{main}
  \end{scriptsize}
\end{frame}

% ----------------------------------------------------------

\end{document}
